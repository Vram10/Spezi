
\documentclass[12pt]{article}
\usepackage[utf8]{inputenc}

\usepackage[T1]{fontenc}
\usepackage[ngerman]{babel}
\usepackage{csquotes}

% documentation: https://ctan.kako-dev.de/macros/latex/contrib/biblatex/doc/biblatex.pdf
\usepackage[backend=biber, style=authoryear]{biblatex}
\addbibresource{references.bib}

% um Grafiken einzubinden, {Angabe des Pfades der Bilder}
\usepackage{graphicx}
\graphicspath{ {images/} }
% Nützliches Online Tool zur Tabellengenerierung: TablesGenerator.com 

% zur korreten Platzierung der Bilder im Titelblatt
\usepackage{chngpage}
\usepackage{calc}
\usepackage{geometry}

% Automatisches Generieren von Hyperlinks bei Verweisen (ref / cite)
\usepackage{hyperref}
\hypersetup{
  colorlinks   = true,    % Colours links instead of ugly boxes
  urlcolor     = blue,    % Colour for external hyperlinks
  linkcolor    = black,    % Colour of internal links
  citecolor    = red      % Colour of citations
}

\usepackage{amssymb}

\begin{document}

    \newgeometry{top=2cm,bottom=2cm}

\begin{titlepage}
    
    \begin{flushright}
        \includegraphics[width=0.3\textwidth]{banner_top}
    \end{flushright}

    \begin{flushleft}
        \vspace*{1cm}
        \Huge
        \textbf{Homomorphe Verschlüsselung – Theorie und Praxis}
        
        \vspace{1.5cm}
        \textbf{Marvin Hahn}\\
        \textbf{Julian van Dyken}
        
        \vspace{0.5cm}
        \Large
        Exposé für die Lehrveranstaltung\\
        Spezielle Verfahren der IT-Sicherheit
        
        \vspace{0.5cm}
        im Studiengang\\
        Informatik
        
    \end{flushleft}        

    \vspace{0.5cm}
    \begin{flushleft}
        Hochschule Emden/Leer\\
        Prof. Dr. Patrick Fehlke\\
        M.Eng. Frederik Gosewehr\\
        \today
    \end{flushleft}    

\end{titlepage}

\restoregeometry
    \newpage

    \pagenumbering{Roman}   
    \tableofcontents
    \newpage

    \pagenumbering{arabic} 
    \section{Einleitung}
        \label{sec:einführung}
Normalerweise muss man verschlüsselte Daten entschlüsseln, um darauf Operationen ausführen zu können. Nutzende müssen daher einen Teil ihrer Privatsphäre einbüßen, damit die Daten verarbeitet werden können. Dies stellt bei sensiblen Daten ein Risiko dar. Mit homomorpher Verschlüsselung gibt es die Möglichkeit, solche Operationen direkt auf den verschlüsselten Daten durchzuführen. Dadurch bleiben sensible Informationen geschützt. Nutzende bekommen das Ergebnis in verschlüsselter Form zurück und können es dann selber entschlüsseln. Es ist also so, als ob die Berechnungen direkt auf den entschlüsselten Daten durchgeführt wurden. Nützlich ist sowas beispielsweise beim Sammeln von Stimmen oder auch bei Cloud-Computing.
    
    \section{Das Ziel der Ausarbeitung} 
        \label{sec:ziel}
    Das Ziel unserer Ausarbeitung ist, die grundlegende Theorie hinter homomorpher Verschlüsselung zu erklären und deren Anwendung in der Praxis. Wir differenzieren zwischen den und erläutern die verschiedenen Arten der homomorphen Verschlüsselung. Dazu werden wir eine strukturierte Übersicht der aktuellen Verfahren erarbeiten und Beispiele für praktische Anwendungen aufzeigen. Diese werden dann von uns auf Basis der aktuellen Forschung – insbesondere hinsichtlich zukünftigen Quantencomputings – bewertet.

    \section{Kern der Arbeit}
        \label{sec:kern}
    Der Kern unserer Arbeit wird auf der Praxis homomorpher Verschlüsselung liegen.    

    \newpage    
    
    \section{Forschungsfragen}
        \label{sec:fragen}
\begin{itemize}
    \item \textbf{Was ist die Kernidee hinter Homomorpher Verschlüsselung?}
    \item \textbf{Welche Arten von homomorpher Verschlüsselung gibt es?}
    \begin{itemize}
        \item Welche Vor- und Nachteile gibt es?
    \end{itemize}
    \item \textbf{Auf welchen mathematischen Grundlagen basieren die Arten?}
    \item \textbf{Wie kann man homomorphe Verschlüsselung in der Praxis einsetzen?}
    \begin{itemize}
        \item Wo ist es eventuell schon im Einsatz?
        \item Was ist in Zukunft noch möglich?
        \item In welchen Gebieten lohnt es sich?
    \end{itemize}
    \item \textbf{Ist homomorphe Verschlüsselung post-quanten sicher?}
\end{itemize}

\vspace{1em}
    
    \section{Theorien}
        \label{sec:praxis}
\begin{itemize}
    \item \textbf{Welche Theorien gibt es?}
    \begin{itemize}
        \item Grundlegende Theorie: partial, somewhat, full
        \item Bootstrapping
    \end{itemize}

    \item \textbf{Wo wollen wir anknüpfen?}
    \begin{itemize}
        \item Zunächst am Beispiel bekannter Methoden (z.B. RSA oder ElGamal) die Theorie erklären
    \end{itemize}
\end{itemize}
    
    \section{Methodisches Vorgehen}
         \label{sec:meth}
\begin{itemize}
	\item \textbf{Literaturrecherche}
	\begin{itemize}
		\item Suche in wissenschaftlichen Datenbanken (z.B. SpringerNature)
	\end{itemize}
\end{itemize}
    \newpage

    \cite{acar2018}
    \cite{khalimov2022}
    \cite{buell2024}
    
    \section{Literatur}
    	\printbibliography
    	
    %\newpage
    

\end{document}