\label{sec:praxis}
\begin{itemize}
    \item \textbf{Welche Theorien gibt es?}
    \begin{itemize}
    	\item \( E(m_1)\bigstar E(m_2) = E(m_1\bigstar m_2),\quad \forall m_1,m_2\in M. \)
        \item Arten: partial (partiell), somewhat (eingeschränkt), full (vollständig)
        \item Squashing
        \item Bootstrapping
    \end{itemize}


	\item \textbf{Beispielhafte Verfahren}
	\begin{itemize}
		
		\item \textit{Partiell homomorphe Verschlüsselung}
		\begin{itemize}
			\item RSA
			\item Goldwasser-Micali
			\item ElGamal
			\item Benaloh
			\item Paillier
		\end{itemize}
	
		\item \textit{Eingeschränkt homomorphe Verschlüsselung}
		\begin{itemize}
			\item BGN
		\end{itemize}
	
		\item \textit{Vollständig homomorphe Verschlüsselung}
		\begin{itemize}
			\item Ideale gitterbasierte Schemen
			\item Schemen über Integer
			\item LWE(Learning With Errors)-basierte Schemen
			\item NTRU-Like FHE Schemes
		\end{itemize}
	
	\end{itemize}

    \item \textbf{Anknüpfungspunkte an bekannten Verfahren}
    \begin{itemize}
        \item RSA
        \item ElGamal
    \end{itemize}

\end{itemize}