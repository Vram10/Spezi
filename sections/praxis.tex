Aufgrund der in der Einführung (\ref{sec:einführung}) beschriebenen Probleme bei Softwareentwicklung finden Versionskontrollsysteme in der Praxis ihren Einsatz

\subsection{Anwendungsfälle}
    Dabei kann man die Systeme nicht nur für ihren ursprünglichen Zweck\footcite[Vgl.][]{10.1371/journal.pcbi.1004668}, das Versionieren von Quellcode in Softwareprojekten nutzen, sondern auch für allmöglichen Dateien, bei denen man eine Historie haben will und die man ggf. als Team erstellt. 

\subsection{Vorhandene Software}
    Beide Arten sind nicht nur theoretisch erklärbar, sondern sind bereits technisch umgesetzt und finden in Unternehmen als Standardsoftware ihren Einsatz. 

    \subsubsection{Centralized}
        Durch das Open-Source-Projekt Concurrent Versions System wurde diese Art erstmalig populär und im Jahr 2000 mit Subversion\footcite[siehe][für eine umfangreiche Einführung]{book:svn} 
        neu implementiert\footcite[siehe][S. xi Anhang B für Unterschiede der Implementierungen]{book:svn}.
    
    \subsubsection{Distributed}
        An dieser Stelle sind vor allem 
        \emph{Git}\footcite[Siehe][für eine umfangreiche Einführung]{pro_git}
        sowie das Konkurenzprodukt
        \emph{Mercurial}\footcite[Siehe][für eine umfangreiche Einführung]{mercurial_guide}
        zu nennen. 
        Beide fallen unter die Kateogrie freie Software\footcite[Definition][S.281-284]{Grassmuck2004} 
        doch unterscheiden sich in Komplexität und Erlernbarkeit \footcite[Vgl.][]{git_vs_mer}