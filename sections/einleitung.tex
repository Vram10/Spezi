\label{sec:einführung}
Normalerweise muss man verschlüsselte Daten entschlüsseln, um darauf Operationen ausführen zu können. Nutzende müssen daher einen Teil ihrer Privatsphäre einbüßen, damit die Daten verarbeitet werden können. Dies stellt bei sensiblen Daten ein Risiko dar. Mit homomorpher Verschlüsselung gibt es die Möglichkeit, solche Operationen direkt auf den verschlüsselten Daten durchzuführen. Dadurch bleiben sensible Informationen geschützt. Nutzende bekommen das Ergebnis in verschlüsselter Form zurück und können es dann selber entschlüsseln. Es ist also so, als ob die Berechnungen direkt auf den entschlüsselten Daten durchgeführt wurden. Nützlich ist sowas beispielsweise beim Sammeln von Stimmen oder auch bei Cloud-Computing.