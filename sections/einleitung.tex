\label{sec:einführung}
Normalerweise muss man verschlüsselte Daten entschlüsseln um damit Operationen durchzuführen z.B. Aufaddieren der Stimmen bei einer Wahl oder Daten für Maschinelles lernen. Der Nutzer muss daher einen Teil seiner Privatsphäre einbüßen, damit diese Verarbeitet werden können. Das birgt Risiken, weil die Informationen wer für wen gestimmt hat sehr sensibel sind und man auch nicht unbedingt Google seine sensiblen Daten für Maschinelles lernen zur Verfügung stellen möchte. Mit Homomorpher Verschlüsselung gibt es die Möglichkeit solche Operationen direkt auf den verschlüsselten Daten durchzuführen. Dadurch bleiben sensible Informationen geschützt. Auf die Beispiele von eben bezogen bedeutet das, dass es möglich ist ein Modell zu haben, was die Daten Homomorph auswertet, dass bedeutet, man gibt seine Informationen verschlüsselt an einen Anbieter, dieser kann diese dann auswerten und Berechnungen ausführen. Das Ergebnis dieser Berechnungen ist dann immer noch verschlüsselt. Dann bekommt man das Ergebnis zurück und kann es einfach entschlüsseln. Es ist also so, als ob die Berechnungen direkt auf dem Entschlüsselten Daten durchgeführt wurden.